




The Internet of Things (IoT) has revolutionized the way we interact with devices. From smart grids, to healthcare and home automation systems. This paradigm shift inadvertently allows for ease of device access. Unfortunately, this technological advancement has been met with unforeseen security challenges. One major challenge is malicious intrusions --A common way of detecting a malicious attack is by treating an attack as an  anomaly and using anomaly detection techniques to pinpoint the source of an intrusion. In a given IoT device, provenance graphs which denotes causality  between system events offers immense benefit for anomaly detection. Provenance provides a comprehensive history of activities performed on a system which indirectly ensures trust. Given a provenance graph, how do we determine if anomalous activities exists? This paper seeks to address this issue. In this paper, we present a lightweight approach to providing anomaly detection in IoT devices using provenance graphs. We introduce an error tolerant graph embedding technique using frequencies of nodes and edges in which provenance graphs are converted into a vector space representation. This vector space representation of the graphs from learning and detection phase is used as an input  parameter for for a similarity measure and clustering algorithm. We evaluate the effectiveness of our approach with IoT device which generates provenance graphs.