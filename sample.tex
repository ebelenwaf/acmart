\documentstyle[epsf]{acmconf}

%----------------- theorem definitions ----------
\newtheorem{Theorem}{Theorem}[section]
\newtheorem{Definition}[Theorem]{Definition}
\newtheorem{Example}{\it Example}[section]

%-------------------- environment definitions -----------------
\newenvironment{example}[0]{\begin{Example} \it}{\end{Example}}
\newenvironment{proof-of}[1]{{\em Proof of #1:}}{}

%----------------- command definitions ----------
\newcommand{\nnn}{{\tt \bf  not}}

%The following was borrowed from the SIGMOD '96 website prepared by
%Inderpal Singh Mumick, Bell Laboratories
% .ps files prepared  in Europe and Asia are for different paper size.
% These .ps files do not print well on paper in the US.  To be able to
% print .ps files prepared in Europe and Asia, we need to shift the
%printing area down by 0.65 inch.  The following command can be used to shift
%the printing area down.  Uncomment the line
%\eurasiatrue
%when preparing the  .ps file to mail to the US.
\newif\ifeurasia
\eurasiafalse
%\eurasiatrue
\ifeurasia
\voffset=.65in
\fi
%

%%%%%%%%%%%%%%%%%%%%%%% END of LaTeX definitions %%%%%%%%%%%%%%%%%%%%%%%%%%%%

\begin{document}
\bibliographystyle{plain}
\title{\LARGE \bf Coherence Approach to Logic Program Revision
\thanks{
This paper is based on the technical report 
{\em Coherence Approach to Logic Program Revision},
ISIS-RR-94-19E, Fujitsu Laboratories, 1994.
A preliminary version of this paper appears in the
Proc. of the 12th International Conference on Logic Programming, Page
167-181, June 1995.
The work of the first author was performed while visiting
the ISIS, Fujitsu Labs, Numazu, Japan.}}
\author{{\bf Li-Yan Yuan} \\
{\small Computing Science} \\
{\small University of Alberta}\\
{\small Edmonton Canada T6G 2H1} \\
{\small yuan@cs.ualberta.ca }\\
\and {\bf Jia-Huai You} \\
{\small Computing Science} \\
{\small University of Alberta} \\
{\small Edmonton Canada T6G 2H1} \\
{\small you@cs.ualberta.ca} \\
}
\maketitle
\begin{abstract} 
We present a new approach to the problem of 
revising extended programs, based on the {\em coherence theory}
initially advocated by Gardenfors for belief revision.
Our approach resolves contradiction by removing only conflicting
information, not the believed source of it, and therefore, keeps
information loss minimal. Furthermore, since there is
no need of searching for problematic assumptions, as is done in the
traditional assumption-removal approach, our approach provides a
skeptical revision semantics that is tractable.

We define the skeptical and credulous coherence semantics
and show that both semantics can be characterized 
in terms of the fixpoint semantics of a revised program using a 
simple program revision technique. 
These semantics provide a suitable framework for knowledge and belief
revision in the context of logic programs.
Semantical properties and advantages 
of the proposed revision semantics are also analyzed.
\end{abstract}

\section{Introduction}
The extension of logic programs with classical negation
significantly  increases the
expressive power of logic programs, but also presents new challenges
\cite{gelfondlifschitz90}.
Among other things, the contradiction problem brought up
by classical negation has to be addressed.

Unlike normal programs, an extended program with classical negation may
not be consistent.
For example, 
$\Pi = \{ \neg a \leftarrow ; \ \
a \leftarrow ; \ \ b \leftarrow\}$ is considered contradictory
since both $a$ and $\neg a$ can be derived from it. 

Many attempts have been made to resolve the contradiction problem.
However, almost all proposals are based on one approach:
Resolve the contradiction problem by removing some problematic 
assumptions 
\footnote{The assumptions here refer to the values of 
the default negations. The term is used simply because 
such values are usually determined by first assuming a value and
then justifying the value.}
, though different mechanisms may be used for the
removal.
Three notable examples are the contradiction removal semantics
\cite{paa91a}, the argumentation semantics \cite{dung93a}, and
the assumption denial semantics \cite{youyuan94}.
The  assumption removal approach, though avoiding contradictory 
conclusions caused by negation as failure, is not suitable for many 
applications, such as knowledge and belief revision.

An analysis of semantical properties shows that 
the assumption removal approach 
violates two important semantical principles in logic 
programming and knowledge revision: 
the principles of {\em conservatism} and {\em relevance}.

One of the basic postulates in knowledge revision is that of 
{\em conservatism} which requires that any conclusions derived from a
revised system be derivable from the system prior to the revision
\cite{gardenfors92}. 
%For essentially the same purpose, 
%Witteveen and Brewka formulate the notion of 
%{\em minimal augmentation} in incremental belief revision 
%in the context of reason maintenance systems \cite{WB93} and show that 
%a violation of this principle may yield
%{\em spurious} beliefs, i.e., the beliefs that cannot be justified,
%in the process of adding additional clauses.
%
The assumption-removal approach, though avoiding contradiction, does not
address the consequences of assumption removal. As such, conservatism may not hold. 
\begin{example}\label{example conservatism}
Consider $\Pi$ below
\[
\neg a \leftarrow \nnn b ; \ \
a \leftarrow ; \ \
c \leftarrow \nnn \neg a,
\]
and the credulous argumentation semantics {\rm \cite{dung93a}}.

With each negative literal $\neg \alpha$ viewed as a named proposition
$\bar{\alpha}$, the credulous argumentation semantics of $\Pi$ 
does not imply $c$, which demonstrates that $c$ is not true with
respect to the credulous argumentation semantics without involving any
revision mechanism. Since $\Pi$ has a unique preferred extension
$N = \{\nnn \neg a\}$,  the credulous argumentation semantics 
of $\Pi$ does imply $a$ and $c$ which violates conservatism (see Section 5 for
details). 
\end{example}
The principle of {\em relevance} requires that the values of the literals 
in a set be determined by only those clauses that define them
\cite{dix92,huyuan91}. Relevance  has been considered essential in 
any goal-directed query evaluation \cite{ap93} while 
the goal-directed theoretical characterization forms suitable bases for
logic programming \cite{mnps91}.
The following example demonstrates that the assumption removal approach
may violate relevance.
\begin{example} Assume $\Pi$ is defined by
\[
a \leftarrow \nnn b ; \ \
\neg a \leftarrow \nnn b ; \ \
d \leftarrow \nnn b.
\]
To resolve the conflict between $a$ and $\neg a$, the
assumption-removal semantics 
has to remove $\nnn b$. Consequently, $d$ cannot be derived.
Note that the relevant program of $d$
, i.e., the set of all clauses that define $d$, 
is $\{d \leftarrow \nnn b\}$ which deduces $d$.
\end{example}
Furthermore, the assumption removal approach suffers computationally from
finding a minimal set of problematic assumptions. This is the cause of 
intractability in the contradiction removal semantics, as shown 
in \cite{wit93a}, and in the grounded version of 
the {\em ideal skeptical semantics}, as shown in \cite{youcart94}.
Although Dung's grounded argumentation semantics \cite{dung93a}
does not attempt to compute
a minimal removal set, it is also NP-hard since 
it is NP-complete to determine if an atom is derivable from a sound 
argument \cite{youcart94}, an essential property in attack and 
counterattack relationships.

In this paper, we present an alternative approach, based on the
coherence theory \cite{gardenfors92}, to removing contradiction
from extended programs. This approach provides a suitable semantics for
knowledge and belief revision in the framework of logic programs. 
The idea is simple but very effective:
\begin{quotation}
\noindent
Given an extended logic program $\Pi$ and a set $N$ of assumptions,
the coherence revision semantics is determined by the disjunction of
maximal consistent sets of literals 
derivable from $\Pi^N$
\footnote{$\Pi^N$ is the program obtained by GL-transformation}
\end{quotation}
Note that the assumption-removal approach is based on the assumption that
inconsistency is caused by problematic assumptions while the coherence
approach removes contradiction regardless the source of it.

We investigate the coherence approach mainly because 
(1) the coherence approach resolves contradiction by removing only
conflicting 
information, not the {\em believed source} of it, and therefore, 
keeps information loss  minimal; and (2) in this approach
there is no need
of searching for minimal sets of problematic assumptions, and thus 
it provides a skeptical revision semantics that is tractable.

In the paper, we first define the notion of consistent-and-justified (CJ)
partial models, based on a negation-as-failure 
rule and the coherence theory, and then present the skeptical and credulous
coherence semantics in terms of CJ partial models. 

A remarkable fact about CJ partial models of a program $\Pi$
is that they are fixpoints of a suitably defined operator for 
program $\Pi_R$, which is obtained from $\Pi$
by a simple program revision technique:
semi-normalizing the {\em doubling} program \cite{wallace90}. 
This program revision technique is simple but not trivial, and it gives
a natural revision semantics based on the coherence theory.

We also analyze the semantical properties of the proposed revision
semantics and show that these semantics satisfy conservatism and 
relevance, among a number of properties along the line similar to that of 
Dix \cite{dix94b}.  


\section{Conclusions}
The coherence approach to logic program revision is based on the idea
that contradiction may be resolved by removing only conflicting
information, without searching for believed source of conflict.
We have demonstrated that this approach satisfies many desirable
properties,
inclosing the conservatism and relevance. Furthermore, simply because
of the fact that the revision can be achieved regardless of the source
of conflict, the coherence approach provides a tractable
skeptical revision semantics.

Various coherence semantics provides an alternative  approach to logic program
revision, and its applications are currently under investigation.

\section*{Acknowledgements}
The authors would like to thank J\"uergen Dix and anonymous reviewers
for providing many constructive comments.  
This work is partially supported by grants from the National
Science and Engineering Research Council of Canada and by
the ISIS, Fujitsu Labs, Numazu, Japan.


%\bibliography{logic}

\begin{thebibliography}{10}

\bibitem{ap93}
J.J. Alferes, C.~V. Damasio, and L.~M. Pereira.
\newblock Top-down query evaluation for well-founded semantics with explicit
  negation.
\newblock In {\em Proc. ECAI '94}, pages 140--144, 1994.

\bibitem{ap93a}
J.J. Alferes, P.~Dung, and L.~Pereira.
\newblock Scenario semantics of extended logic programs.

\bibitem{dix92}
J.~Dix.
\newblock A framework for representing and characterizing semantics of logic
  programs.
\newblock In {\em Proceedings of the KR'92}, pages 591--602, 1992.

\bibitem{dix94b}
J.~Dix.
\newblock A classification theory of semantics of normal logic programs: {II}.
  weak properties.
\newblock {\em to appear}, 1994.

\bibitem{dung93a}
P.~M. Dung.
\newblock An argumentation semantics for logic programming with explicit
  negation.
\newblock In {\em Proc. the 10th ICLP % 10th International Conference on Logic
  Programming}, pages 615--630. The MIT Press, 1993.

\bibitem{gardenfors92}
Peter Gardenfors, editor.
\newblock {\em Belief Revision}.
\newblock Cambridge University Press, 1992.

\bibitem{gelfondlifschitz90}
M.~Gelfond and V.~Lifschitz.
\newblock Logical programs with classical negation.
\newblock In {\em Proceedings of the 7th International Conference on Logic
  Programming}, pages 579--597. The MIT Press, 1990.

\bibitem{huyuan91}
Y.~Hu and L.~Y. Yuan.
\newblock Extended well founded semantics for logic programs with negations.
\newblock In {\em Proceedings of the 8th International Conference on Logic
  Programming}, pages 412--425. The MIT Press, 1991.

\bibitem{mnps91}
D.~Miller, G.~Nadathur, F.~Pfenning, and A.~Scedrov.
\newblock Uniform proofs as a foundation for logic programming.
\newblock {\em Annals of Pure and Applied Logic}, 51:125--157, 1991.

\bibitem{paa91a}
L.M. Pereira, J.J. Alferes, and J.N. Aparicio.
\newblock Contradiction removal within well founded semantics.
\newblock In {\em Proc. of the 1st International Workshop on Logic Programming
  and Nonmonotonic Reasoning}, pages 106--119, 1991.

\bibitem{wallace90}
M.~Wallace.
\newblock Unrestricted logic programs or if stratification is the cure, what is
  the malady?
\newblock In {\em Proc. of European Conference on AI}, pages 682--687, 1990.

\bibitem{wit93a}
C.~Witteveen and C.~M. Jonker.
\newblock Revision by expansion in logic programs.

\bibitem{youcart94}
J.H. You and R.~Cartwright.
\newblock Tractable argumentation semantics via iterative belief revision.
\newblock In {\em Proc. of ILPS'94}, pages 239--254, 1994.

\bibitem{youyuan94}
J.H. You and L.Y. Yuan.
\newblock Logic programming with assumption denials.
\newblock In {\em Proc. of the Workshop 5, ICLP'94}, 1994.

\end{thebibliography}
\end{document}
